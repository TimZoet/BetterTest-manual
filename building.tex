\code{BetterTest} requires a compiler that supports C++20. It has been tested on:

\begin{itemize}
	\item Windows 10 with MSVC 19.28.
	\item Ubuntu 20.04 with GCC 11.
\end{itemize}

\subsection{Dependencies}

To build \code{BetterTest} you must have several packages installed. On Windows, you could use \href{https://github.com/microsoft/vcpkg}{vcpkg} to install them. On Linux, \href{https://conan.io}{Conan}. Not all packages are required for all configurations. \code{BetterTest} is guaranteed to work with the following versions of these packages:

\begin{itemize}
	\item nlohmann/json 3.9.1\cite{nlohmann}. Always required.
\end{itemize}

\subsection{Getting The Code}
\label{section:build:get}

\code{BetterTest} can be cloned from GitHub. Note that since the repository contains submodules, you have to pass an additional parameter. See Listing~\ref{lst:build:clone}.

\lstinputlisting[caption={Cloning.}, label={lst:build:clone}, language=sh]{snippets/clone.cmd}

\subsection{Configuration}
\label{section:build:config}

There are currently no CMake configuration variables specific to \code{BetterTest}.

\subsection{Windows}
\label{section:build:windows}

Assuming you just cloned the repository to a folder named source, you can configure and build using the commands shown in Listing~\ref{lst:build:windows}.

\lstinputlisting[caption={Building on Windows using MSVC and vcpkg.}, label={lst:build:windows}, language=sh]{snippets/build_windows.cmd}

\subsection{Linux}
\label{section:build:linux}

Assuming you just cloned the repository to a folder named source, you can configure and build using the commands shown in Listing~\ref{lst:build:linux}.

\lstinputlisting[caption={Building on Linux using GCC and Conan.}, label={lst:build:linux}, language=sh]{snippets/build_linux.cmd}
