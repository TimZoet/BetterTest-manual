\subsection{Creating a Test Suite}
\label{section:guide:testsuite}

A test suite corresponds directly to an executable. This executable should be linked to the \code{bettertest} library.

\subsection{A First Test}
\label{section:guide:firsttest}

To create a new test, we must define a class deriving from \code{bt::UnitTest}. This base class takes at least one template parameter, namely the deriving class. Additionally we can pass any number of mixin classes. These mixin classes provide the actual testing methods. In this example, we're going to use just the \code{CompareMixin} class.

Tests are invoked through \code{operator()}, so we have to override it. Inside of the function we're going to call 3 of the methods provided by \code{CompareMixin} to perform the following comparisons: \code{42 == 42}, \code{10 <= 20} and  \code{0 <= 200 <= 100}. Obviously, that last check will fail.

Inside of the main function, all we need to do is call \code{bt::run}. To this function we pass all our test classes as template parameters. It also takes the command line arguments (\code{argc} and \code{argv}) and a string, which is used as the test suite name.

\lstinputlisting[caption={Defining and running a simple unit test.}, label={lst:guide:firsttest}]{snippets/first_test.cpp}

\subsection{Running the Test Suite}
\label{section:guide:run}

Running the test suite is as simple as launching the executable. Depending on the selected output format, one or more files are generated. These \marginlabel{See Section~\ref{section:output} for output formats.}files will contain the test results, specifying for each test and check whether or not they succeeded and, if not, why.

To run specific tests, run tests in parallel, specify an output format and more, various \marginlabel{See Section~\ref{section:cmd} for a detailed description of all arguments.}command line arguments can be passed.

Running the test suite specified above, may result in the output shown in Listing~\ref{lst:guide:runoutput}.

\lstinputlisting[caption={Example output.}, language={bash}, label={lst:guide:runoutput}]{snippets/run_output.cmd}

\subsection{More Details}

The name of a test is determined automatically from the class name. Any namespaces are prepended without \code{::}. In case you wish to explicitly specify the name, which can be useful when e.g. the class or namespace names are subject to change, this can be done by specifying a static class member. See Listing~\ref{lst:guide:explicitname}.

\lstinputlisting[caption={Specifying an explicit test name.}, label={lst:guide:explicitname}]{snippets/explicit_test_name.cpp}

The methods of all built-in \marginlabel{See Section~\ref{section:mixins} for a full spec of the built-in mixin classes.}mixin classes return an object of type \code{CheckResult}. This object can be used to supply extra information when a check failed or even terminate the entire test. Termination is sometimes required, for example to prevent out-of-bounds access or dereferencing null pointers. See Listing~\ref{lst:guide:checkresult}.

\lstinputlisting[caption={Terminating a test after a failed check.}, label={lst:guide:checkresult}]{snippets/check_result.cpp}