\code{BetterTest} is a modern, macro free, extensible testing framework.

Instead of having free functions wrapped in macros, a unit test is defined using a class. When invoked, this class can (should) perform all sorts of checks. These checks too are not implemented using macros, but grouped in various \marginlabel{See Section~\ref{section:mixins}.}mixin classes from which your unit test classes can inherit. Not using macros has one downside: as long as C++ does not have compile time reflection, it is necessary to manually register them. A small price to pay...

There are other testing frameworks that have some odd quirks, which are mostly resolved in \code{BetterTest}:

Sometimes a failed check should immediately terminate a test, for instance because something unexpectedly returned a null pointer. Instead of having two different methods to differentiate between terminating and non-terminating failures, there is always just one. For example, there are no \code{expectEQ} and \code{requireEQ}, but only a \code{compareEQ}. The return value of these methods can be used to terminate or print a warning message.

Also, if e.g. two values that are being compared cannot be converted to a string, this will never prevent compilation. \code{BetterTest} will do its best to look for a conversion, but if no such conversion exists it will write a placeholder to the result files.

Finally, tests can be run in parallel out of the box. No need to write silly scripts that simulate this by running the same executable multiple times with different parameters.

By default, \code{BetterTest} can export test results as JSON or XML. The format of the generated files is \marginlabel{See Section~\ref{section:output}.}specified very precisely, so integrating \code{BetterTest} into some automated build pipeline is perfectly possible.

(Almost) all aspects of \code{BetterTest} are extensible. New checks can be defined by implementing new \marginlabel{See Section~\ref{section:mixins:custom}.}mixin classes. \marginlabel{See Section~\ref{section:output:custom}.}Custom output formats can be created using custom importers and exporters.

For build instructions, see Section~\ref{section:build}. For an introductory guide, see Section~\ref{section:guide}.

\subsection{Acknowledgements}
\label{section:introduction:acknowledgements}

\code{BetterTest} currently relies on the following libraries:

\begin{itemize}
	\item nlohmann/json\cite{nlohmann}.
	\item pugixml\cite{pugixml}.
	\item hhdate\cite{hhdate}.
\end{itemize}