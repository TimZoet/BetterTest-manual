As stated in Section~\ref{section:guide}, all testing methods are provided through mixin classes. \code{BetterTest} comes with a number of these classes, and adding more yourself is easy.

For each built-in mixin class, a short description is given below, as well as a full specification of the \marginlabel{See Section~\ref{section:output} for an overview of all output.}generated output. Additionally, the final subsection explains how to implement a custom mixin class.

\subsection{Compare Mixin}
\label{section:mixins:compare}

The \code{CompareMixin} class provides a number of methods to compare (ranges of) values. It provides the usual comparison function: equality (\code{compareEQ}), less than (\code{compareLT}), greater or equal (\code{compareGE}), near equality (\code{compareClose}), etc.

\subsubsection{Output}

All results of the \code{CompareMixin} can be found in a group named \code{"compare"}.

Each comparison is added to the list of \code{results}. At a minimum the \code{location} (both the full path to the file as well as the position in the file) and \code{status} (either \code{"success"} or \code{"failure"}) can be found. In case a comparison failed, the optional \code{error} value will contain a non-empty string with the relevant values (assuming they could be converted to a string) and how they were compared. For example, \code{compareEQ(10, 20)} would result in \code{"10 == 20"}.

The \code{stats} contain aggregate results, listing the \code{total} number of checks and their outcome: \code{failure}, \code{success} or, if an unexpected exception was thrown, \code{exception}. The sum of these three values should obviously be equal to \code{total}. If \code{total} equals \code{success}, the test passed.

\lstinputlisting[caption={Compare Mixin output.}, label={lst:mixins:compare:output}]{snippets/compare_mixin.json}

\subsection{Exception Mixin}
\label{section:mixins:exception}

The \code{ExceptionMixin} class provides a number of methods to assert that code does (\code{expectThrow}) or does not throw (\code{expectNoThrow}).

\subsubsection{Output}

The output of the \code{ExceptionMixin} is almost identical to that of the \code{CompareMixin}. The only differences are that it is of course stored in a group named \code{"exception"} and there is no \code{exception} variable in the aggregate results.

\lstinputlisting[caption={Exception Mixin output.}, label={lst:mixins:exception:output}]{snippets/exception_mixin.json}

\subsection{Custom Mixins}
\label{section:mixins:custom}

New mixins can be created my implementing the \code{IMixin} interface. This interface declares a number of methods that must be overridden. The built-in mixins implement this interface as well, and are perhaps the best example of how to do this.

First and foremost, the \code{getName} method should return a unique string by which the mixin can be identified. Also of importance is the \code{isPassing} method. This is called just after the test is run, and should return whether or not it passed. If the mixin should write something to the console, the \code{out} member variable can be used. Writing to this output stream is thread safe.

The \code{IMixin} interface also declares two methods to exports results to either JSON or XML. These can be optionally overridden, depending on whether or not you actually care for using that output format. All values that are written to the JSON or XML object should be inside of a group with the mixin name. Again, refer to the existing mixin classes to get a better idea of how this should be done. For custom output formats, please refer to Section~\ref{section:output:custom}.
